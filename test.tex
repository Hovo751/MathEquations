\documentclass{article}
\usepackage{fontspec}
\setmainfont{DejaVu Serif}

%%%%%%%%%%%%%%%%%%%%%%%%%%%%%%%%%%%%%%%%%%%%%%%%%%%%%%%%%%%%%%%%%%%%%%%%%%%%%%%%%%%%%%
\usepackage{tikz, dsfont}
%%%%%%%%%%%%%%%%%%%%%%%%%%%%%%%%%%%%%%%%%%%%%%%%%%%%%%%%%%%%%%%%%%%%%%%%%%%%%%%%%%%%%%


\usepackage{amsfonts,amsmath,amstext,amsthm,amssymb,amsxtra,mathrsfs, amscd}
\usepackage[top=1in, bottom=1in, left=0.5in, right=0.5in] {geometry}

\newcommand{\seleng}{\selectlanguage{english}}
\usepackage{mathtools}
\mathtoolsset{showonlyrefs,showmanualtags} 
\usepackage{graphicx}
\usepackage{tcolorbox}
\tcbuselibrary{skins}
\renewcommand{\baselinestretch}{1}
\usepackage{setspace}
\definecolor{mycolor}{HTML}{154f6b}


\usepackage{setspace}
\setstretch{1.1}

\usepackage{hyperref} %,hypertexnames=false,colorlinks,[pagebackref]
\hypersetup{
%    bookmarks=true,         % show bookmarks bar?
%    unicode=false,          % non-Latin characters in AcrobatÕs bookmarks
%    pdftoolbar=true,        % show AcrobatÕs toolbar?
%    pdfmenubar=true,        % show AcrobatÕs menu?
%    pdffitwindow=false,     % window fit to page when opened
%    pdfstartview={FitP},    % fits the width of the page to the window
%    pdftitle={My title},    % title
%    pdfauthor={Author},     % author
%    pdfsubject={Subject},   % subject of the document
%    pdfcreator={Creator},   % creator of the document
%    pdfproducer={Producer}, % producer of the document
%    pdfkeywords={keywords}, % list of keywords
%    pdfnewwindow=true,      % links in new window
colorlinks=true,       % false: boxed links; true: colored links
linkcolor=blue,          % color of internal links
citecolor=magenta,        % color of links to bibliography
filecolor=magenta,      % color of file links
urlcolor=cyan           % color of external links
%    pagebackref=true
}



\newtheorem{cor}{Հատկություն}[section]
\newtheorem{problem}{Խնդիր}
\newtheorem{definition}{Սահմանում}[section]
\newtheorem{examp}{\textcolor{RoyalBlue}{Օրինակ}}[section]
\newtheorem{theorem}{Թեորեմ}[section]
\newtheorem{corollary}{\textcolor{Crimson}{Հետևանք}}[section]
\newtheorem{lemma}{\textcolor{Red}{Լեմմա}}[section]
\def\ZR{\ensuremath{\mathbb R}}
\def\BV{\ensuremath{\mathbb BV}}
\def\ZZ{\ensuremath{\mathbb Z}}
\def\ZM{\ensuremath{\mathfrak{M}}}
\def\ZF{\ensuremath{\mathfrak{F}}}
\def\ZB{\ensuremath{\mathfrak{B}}}
\def\ZC{\ensuremath{\mathbb C}}
\def\ZN{\ensuremath{\mathbb N}}
\def\ZT{\ensuremath{\mathbb T}}
\def\ZI{\ensuremath{\mathbb I}}
\def\SAI{\ensuremath{\mathrm {SAI}}}
\def\DFT{\ensuremath{\mathrm {DFT}}}
\def\IDFT{\ensuremath{\mathrm {IDFT}}}
\def\zP{\ensuremath{\mathcal P}}
\def\ZG{{\mathcal G\,}}
\def\|{\ensuremath{||}}
\def\supp{{\rm supp\,}}
\def\md#1#2\emd{\ifx0#1
\begin{equation*} #2 \end{equation*}\fi  %  single line display, no number
\ifx1#1\begin{equation}#2\end{equation}\fi   % single line display, number
\ifx2#1\begin{align*}#2\end{align*}\fi   % aligned display, no number
\ifx3#1\begin{align}#2\end{align}\fi    % aligned display, number
\ifx4#1\begin{gather*}#2\end{gather*}\fi  % multiline, not align, no number
\ifx5#1\begin{gather}#2\end{gather}\fi   % multinline, not align
\ifx6#1\begin{multline*}#2\end{multline*}\fi  %  display too long for one line
\ifx7#1\begin{multline}#2\end{multline}\fi  % as above, with numbers
}
\newcommand {\e }[1]{(\ref{#1})}
\newcommand {\lem }[1]{Լեմմա \ref{#1}}
\newcommand {\trm }[1]{Թեորեմ \ref{#1}}

\renewcommand*{\proofname}{\bfseries Ապացույց}
\numberwithin{equation}

\begin{document}

\section{Անհավասարություններ}

\textbf {Օգտագործվող անհավասարություններ.}

\begin{align}
\frac{a_1+a_2+\cdots +a_n}{n} \ge \sqrt[n]{a_1a_2\ldots a_n}\quad  \text{(Կոշու անհավասարություն) }\label{C} 
\end{align}
\begin{align}
\frac{a_1^2}{b_1}+\frac{a_2^2}{b_2}\cdots \frac{a_n^2}{b_n} \ge \frac{(a_1+a_2+\cdots a_n)^2}{b_1+b_2+ \cdots b_n}  \quad \text{(Կոշու Շվարցի անհավասարություն)} \label{KSh}
\end{align}
\begin{align}
\frac{a_1^3}{b_1c_1}+\frac{a_2^3}{b_2c_2}\cdots \frac{a_n^3}{b_nc_n} \ge \frac{(a_1+a_2+\cdots a_n)^3}{(b_1+b_2+ \cdots b_n)(c_1+c_2+ \cdots c_n)}  \quad \text{(Կոշու Շվարցի անհավասարություն$^3$)} \label{KSh3}
\end{align}
\begin{problem}
Դիցուք $a,b,c$ դրական թվեր են ։ Ապացուցել անհավասարությունը.
\begin{equation}\label{main}
 \frac{a}{b+c}+\frac{b}{a+c}+\frac{c}{b+c} \ge \frac{3}{2}    
\end{equation}

\end{problem}
\textbf {Ապացույց 1.}
Նկատենք, որ \eqref{main} անհավասարությունը համասեռ է, հետևաբար առանց ընդհանրությունը խախտելու կարելի է ենթադրել, որ $a+b+c=1$:


\textbf{Ապացույց 2.}
Նշանակենք $2x=b+c$, $2y=a+c$, $2z=a+b$, այստեղից $a=y+z-x$, $b=x+z-y$, $c=x+y-z$: Տեղադրենք անհավասարության ձախ մասում և կատարենք ձևափոխություն։
$$
\frac{y+z-x}{2x}+\frac{x+z-y}{2y}+\frac{x+y-z}{2z}=\frac{1}{2}(\frac{y}{x}+\frac{z}{x}+\frac{x}{y}+\frac{z}{y}+\frac{x}{z}+\frac{y}{z}-3)
$$
Օգտագործենք Կոշու անհավասարությունը\eqref{C} $n=6$-ի համար։

\begin{align}
\frac{1}{2}(\frac{y}{x}+\frac{z}{x}+\frac{x}{y}+\frac{z}{y}+\frac{x}{z}+\frac{y}{z}-3) \ge \frac{1}{2}(6\sqrt[6]{\frac{y}{x}\frac{z}{x}\frac{x}{y}\frac{z}{y}\frac{x}{z}\frac{y}{z}}-3)=\frac{1}{2}(6-3)=\frac{3}{2}
\end{align}


\textbf{Ապացույց 3.}
Կատարենք հետևյալ ձևափոխությունը:
\begin{equation}
 \frac{a}{b+c}+\frac{b}{a+c}+\frac{c}{b+c}=\frac{a^2}{a(b+c)}+\frac{b^2}{b(a+c)}+\frac{c^2}{c(a+b)}    
\end{equation}
Օգտագործենք Կոշու Շվարցի\eqref{KSh} անհավասարությունը:
\begin{equation}
\frac{a^2}{a(b+c)}+\frac{b^2}{b(a+c)}+\frac{c^2}{c(a+b)} \ge \frac{(a+b+c)^2}{2(ab+bc+ac)}  
\end{equation}
Օգտվենք $a^2+b^2+c^2 \ge ab+bc+ac$ անհավասարությունից:
\begin{equation}
\frac{(a+b+c)^2}{2(ab+bc+ac)}  \ge \frac{a^2+b^2+c^2+2ab+2ac+2bc}{2(ab+bc+ac)} \ge \frac{3(ab+bc+ac)}{2(ab+bc+ac)}=\frac{3}{2}
\end{equation}
\textbf{Կարգավորված հաջորդականություններ}

Ապացուցել, որ եթե $a_1\ge a_2 \ge \cdots\ge a_n$ և $b_1 \ge b_2 \ge\cdots \ge b_n$, ապա $a$-երից և $b$-երից կազմված զույգ առ
զույգ $n$ արտադրյալների գումարը կընդունի մեծագույն արժեքը, երբ յուրաքանչյուր $a_k$ զույգավորվի
$b_k$-ի հետ: Այսինքն ամենամեծ հնարավոր գումարը կլինի՝ $a_1b_1 + a_2b_2 + ⋯ + a_nb_n$(մասնավորապես $a^2+b^2+c^2\ge ab+bc+ac$):

\textbf{Ապացույց.}
Օգտվենք մաթեմատիկական ինդուկցիայից։ $n=2$ դեպքը ակնհայտ է($a_1b_1+a_2b_2\ge a_1b_2+a_2b_1$)։ Ենթադրենք $n=k-1$ դեպքը բավարարում  է, ապացուցենք, որ $n=k$ դեպքը նույնպես բավարարում է։
Նախ ապացուցենք, որ մեծագույն գումարի մեջ $a_1$-ը պետք է լինի $b_1$-ի զույգը։ Ենթադրենք հակառակը. M գումարը մեծագույնն է և նրա մեջ $a_1$-ը զույգ է կազմում որևէ $b_i$-ի հետ, իսկ $b_1$-ը՝ $a_j$-ի հետ։ Քանի որ $n=2$ դեպքում պնդումը ճիշտ է, և $a_1\ge a_j$, $b_1\ge b_j$ \Rightarrow $a_1b_1+a_jb_i\ge a_1b_i+a_jb_1$։ Հետևաբար, եթե $a_1$ և $a_j$ փոխվեն զույգերով նոր գումարը կլինի ավելի մեծ քան M-ը: Ստացանք հակասություն \Rightarrow մեծագույնի դեպքում $a_1$-ը  $b_1$-ի զույգն է։ Այժմ նկատենք, որ, քանի որ $a_2\ge a_3\ge \ldots \ge a_k $, $b_2\ge b_3\ge \ldots \ge b_k $ և $n=k-1$, ապա զույգ առ զույգ արտադրյալների մեծագույն գումարը այս դեպքում կլինի $a_2b_2+a_3b_3+\ldots a_kb_k$-ը \Rightarrow $n=k$-ի մեծագույն գումարի արժեքը կլինի $a_1b_1 + a_2b_2 + ⋯ + a_nb_n$-ը քանի որ արդեն նշեցինք որ $a_1$-ը կզույգվի $b_1$-ի հետ։

\begin{problem}
Դիցուք $a,b,c$ դրական թվեր են ։ Գտնել փոքրագույն արժեքը.
\begin{equation}
 a^5+b^5+c^5    
\end{equation}
եթե
\begin{equation}\label{a+b+c}
a+b+c=3 
\end{equation}
\end{problem}
\textbf{Ապացույց 1.}
Նկատենք, որ օգտագործելով Կոշու անհավասարությունը\eqref{C} $a^5+1+1+1+1$-ի վրա $n=5$-ի համար կստանանք, որ։
\begin{equation}
a^5+1+1+1+1 \ge 5\sqrt[5]{a^5} = 5a
\end{equation}
Հետևություն, որ։
\begin{equation}
a^5+b^5+c^5 \ge 5a+5b+5c-12=5(a+b+c)-12
\end{equation}
Տեղադրենք $a+b+c$-ի տեղը 3\eqref{a+b+c}։
\begin{align}
5(a+b+c)-12=5(3)-12=15-12=3
\end{align}


\textbf{Ապացույց 2.}
Կատարենք հետևյալ ձևափոխությունը:
\begin{equation}
 a^5+b^5+c^5=\frac{a^6}{a}+\frac{b^6}{b}+\frac{c^6}{c}
\end{equation}
Օգտագործենք Կոշու Շվարցի\eqref{KSh3} անհավասարությունը:
\begin{equation}
 \frac{a^6}{a}+\frac{b^6}{b}+\frac{c^6}{c}=\frac{a^6}{1a}+\frac{b^6}{1b}+\frac{c^6}{1c}\ge\frac{(a^2+b^2+c^2)^3}{(a+b+c)(1+1+1)} = \frac{(a^2+b^2+c^2)^3}{3(a+b+c)}
\end{equation}
Օգտագործենք Կոշու Շվարցի\eqref{KSh} անհավասարությունը:
\begin{equation}
 \frac{(a^2+b^2+c^2)^3}{3(a+b+c)}=\frac{(\frac{a^2}{1}+\frac{b^2}{1}+\frac{c^2}{1})^3}{3(a+b+c)}\ge\frac{(\frac{(a+b+c)^2}{1+1+1})^3}{3(a+b+c)}=\frac{(\frac{(a+b+c)^6}{3^3})}{3(a+b+c)}
\end{equation}
Տեղադրենք $a+b+c$-ի տեղը 3\eqref{a+b+c}։
\begin{equation}
 \frac{(\frac{(a+b+c)^6}{3^3})}{3(a+b+c)}=\frac{(\frac{(3)^6}{3^3})}{3(3)}=\frac{3^6}{3^5}=3
\end{equation}

\begin{problem}
Դիցուք $a,b,c$ դրական թվեր են ։ Գտնել փոքրագույն արժեքը.
\begin{equation}
 a^5+b^5+c^5    
\end{equation}
եթե
\begin{equation}\label{a^2+b^2+c^2}
a^2+b^2+c^2=3 
\end{equation}
\end{problem}
Նախ ապացուցենք, որ$3\ge a+b+c$ օգտագործելով Կոշու Շվարցի\eqref{KSh} անհավասարությունը և $a^2+b^2+c^2=3$ \eqref{a^2+b^2+c^2} հավասարությունը:
\begin{equation}\label{a+b+c<=3}
3^2=3(a^2+b^2+c^2)=3(\frac{a^2}{1}+\frac{b^2}{1}+\frac{c^2}{1})\ge 3(\frac{(a+b+c)^2}{1+1+1})=(a+b+c)^2\Rightarrow3\ge a+b+c
\end{equation}
\textbf{Ապացույց 1.}
Նկատենք, որ օգտագործելով Կոշու անհավասարությունը\eqref{C} $a^5+a+1$-ի վրա $n=3$-ի համար կստանանք, որ։
\begin{equation}
a^5+a+1 \ge 3\sqrt[3]{a^6} = 3a^2
\end{equation}
Հետևություն, որ։
\begin{equation}
a^5+b^5+c^5 \ge 3a^2+3b^2+3c^2-3-(a+b+c)=3(a^2+b^2+c^2)-3-(a+b+c)
\end{equation}
Տեղադրենք $a+b+c$-ի և $a^2+b^2+c^2$-ի տեղը 3\eqref{a+b+c}։
\begin{align}
3(a^2+b^2+c^2)-3-(a+b+c) \ge 3(a^2+b^2+c^2)-3-(3)=3(3)-3-3=3
\end{align}


\textbf{Ապացույց 2.}
Կատարենք հետևյալ ձևափոխությունը:
\begin{equation}
 a^5+b^5+c^5=\frac{a^6}{a}+\frac{b^6}{b}+\frac{c^6}{c}
\end{equation}
Օգտագործենք Կոշու Շվարցի\eqref{KSh3} անհավասարությունը:
\begin{equation}
 \frac{a^6}{a}+\frac{b^6}{b}+\frac{c^6}{c}=\frac{a^6}{1a}+\frac{b^6}{1b}+\frac{c^6}{1c}\ge\frac{(a^2+b^2+c^2)^3}{(a+b+c)(1+1+1)} = \frac{(a^2+b^2+c^2)^3}{3(a+b+c)}
\end{equation}
Տեղադրենք $a+b+c$-ի և $a^2+b^2+c^2$-ի տեղը 3\eqref{a+b+c}։
\begin{equation}
 \frac{(a^2+b^2+c^2)^3}{3(a+b+c)}=\frac{(3)^3}{3(a+b+c)} \ge \frac{(3)^3}{3(3)}=3
\end{equation}
\end{document}